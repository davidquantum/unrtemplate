%%%%%%%%%%%%%%%%%%%%%%%%%%%%%%%%%%%%%%%%%%%%%%%%%%%%%%%%%%%%%%%%%%%%%%%%%%%%%%%%%%%%%%%%%%%
%                                     MSthesis v1.1                                       %
%                            By Quan Zou <quan.zou@gmail.com>                             %
%                            Version 1.1 released 01/10/2010                              %
%%%%%%%%%%%%%%%%%%%%%%%%%%%%%%%%%%%%%%%%%%%%%%%%%%%%%%%%%%%%%%%%%%%%%%%%%%%%%%%%%%%%%%%%%%%
%
\begin{abstracts}			% this creates the heading for the abstract page
%\begin{abstractslong}			% if you have a long abstract, use this environment
%\begin{abstractseparate}		% abstract including title and author. etc.

Abstractseparate environment creates a page with the abstract on including title and author 
etc.  It may be required to be handed in separately. If this is not so, then use the other 
2 environments: abstracts and abstractslong. The only difference beween them is 
abstractslong will incooperate more than abstracts does. \\

\noindent {\bf Keywords}: UNR, \LaTeX, thesis template, Mathematics, Statistics. \\

%\end{abstractseparate}
%\end{abstractslong}
\end{abstracts}

%------------------------------------------------------------------------------------------

%%% Local Variables: 
%%% mode: latex
%%% TeX-master: "../thesis"
%%% End: 
